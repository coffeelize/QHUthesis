% !TEX encoding = UTF-8
%%=====自定义页眉页脚
%\pagestyle{fancy}
%\fancyhf{}
%\chead{这里是标题这里是标题这里是标题}

\renewcommand{\headrulewidth}{0.4pt} 
\chapter{青海大学 \LaTeX 模板}
\section{Why \LaTeX ? }
\LaTeX(LATEX,音译“拉泰赫”)是一种基于TEX的排版系统,由美国计算机学家莱斯利·兰伯特(Leslie Lamport)在20世纪80年代初期开发,利用这种格式,即使使用者没有排版和程序设计的知识也可以充分发挥由\TeX{}所提供的强大功能,能在几天,甚至几小时内生成很多具有书籍质量的印刷品。对于生成复杂表格和数学公式,这一点表现得尤为突出。因此它非常适用于生成高印刷质量的科技和数学类文档。这个系统同样适用于生成从简单的信件到完整书籍的所有其他种类的文档。

为了方便青海大学本科生将更多的时间集中于论文的内容当中,而不是在格式的调节上浪费时间。\LaTeX 提供了一个很好的方式。\LaTeX 具有很多优点就不说了,大家可以多用用。
\par 下文就是简单的版式,与青海大学毕业论文写作规范.doc中要求一致。若有不同请与我联系。

bla bla bla bla bla bla bla bla bla bla bla bla bla bla bla bla bla bla bla bla bla bla bla bla bla bla bla bla bla bla bla bla bla bla bla bla bla bla bla bla bla bla bla bla bla bla bla bla bla bla bla bla bla bla bla bla bla bla bla bla bla bla bla bla bla bla bla bla bla bla bla bla bla bla bla bla bla bla bla bla bla bla bla bla bla bla bla bla bla bla

\subsection{选题背景与意义(四号仿宋)}
主要介绍毕业设计选题的背景及为什么要做这个事,即这个事的意义。为了便于操作,可以先把文字复制到写字板后,再粘贴到相应位置,格式会保持不变。
\subsection{国内外研究现状}
主要介绍国内与国外在这个选题方面的研究情况、进展及存在的问题,也就顺势引入你做这个选题的意义了(可以解决存在的问题)。这部分可以分国内、国外两个小节来介绍,也可以不分小节来讲。这部分引用别人的研究成果的较多,一定要进行标注,并在参考文献中按出现顺序进行一一列举出来。凡引用、转述、参考他人的成果或资料,均须注明出处。
\subsubsection{国内研究现状}
主要是国内在这方面的研究情况、进展与存在的问题。
\subsubsection{国外研究现状}
主要是国外在这方面的研究情况、进展与存在的问题。
