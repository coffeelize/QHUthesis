% !TEX encoding = UTF-8

\chapter{模板使用}
\section{模板文件结构\label{sec:files}}
整个模板根目录的文件列表如下:
\begin{center}
\begin{tabular}{|l|l|l|}
\hline
QHUthesis.cls & ---QHUthesis宏包 & \textcolor{red}{{*}}\\
\hline
QHU.cfg & ---QHU宏包配置文件 & \textcolor{red}{{*}}\\
\hline
QHUbib.bst & ---引文样式文件 & \textcolor{red}{{*}}\\
\hline
references/reference.bib & ---bib数据库 & \textcolor{red}{{*}}\\
\hline
figures/QHUbmp.bmp & ---青海大学校名标准字 & \textcolor{red}{{*}}\\
\hline
QHUbachelor.tex & ---\TeX{}样例文件 &\textcolor{red}{{*}}\\
\hline
\end{tabular}
\end{center}
注: \textcolor{red}{{*}} 表示\LaTeX{}模板必须的文件。
\section{示例}
对于论文中最常使用的一些功能在本节中给出示例。
\subsection{公式}
\begin{equation}
\hat{H}=\frac{\epsilon}{2}\hat{\sigma}_{z}-\frac{\Delta}{2}\hat{\sigma}_{x}+\sum_{k}\omega_{k}\hat{b}_{k}^{\dagger}\hat{b}_{k}+\sum_{k}\frac{g_{k}}{2}\hat{\sigma}_{z}(\hat{b}_{k}+\hat{b}_{k}^{\dagger})\label{eq:sbm}
\end{equation}
根据公式\ref{eq:sbm}可知,这个是对公示的引用。
\begin{align}\label{eqn3:9}
\int_{-\infty}^{+\infty}S(\tau,f)\,\mathrm{d}\tau&=\int_{-\infty}^{+\infty}x(t)\left\{\int_{-\infty}^{+\infty}\frac{|f|}{\sqrt{2\pi}}e^{\frac{-|f|^2(\tau-t)^2}{2}}\,\mathrm{d}\tau\right\}e^{-j2\pi ft}\,\mathrm{d}t\notag\\
&=\int_{-\infty}^{+\infty}x(t)e^{-j2\pi ft}\left\{\int_{-\infty}^{+\infty}\frac{1}{\sqrt{\pi}}e^{-\left[\frac{|f|(\tau-t)}{\sqrt{2}}\right]^2}\,\mathrm{d}\frac{|f|(\tau-t)}{\sqrt{2}}\right\}\,\mathrm{d}t
\end{align}
令$\theta=\frac{|f|(\tau-t)}{\sqrt{2}}$,则式\eqref{eqn3:9}可改写为
\begin{align}\label{eqn3:10}
\int_{-\infty}^{+\infty}S(\tau,f)\,\mathrm{d}\tau&=\int_{-\infty}^{+\infty}x(t)e^{-j2\pi ft}\,\mathrm{d}t\frac{1}{\sqrt{\pi}}\int_{-\infty}^{+\infty}e^{-\theta^2}\,\mathrm{d}\theta\notag\\
&=\int_{-\infty}^{+\infty}x(t)e^{-j2\pi ft}\,\mathrm{d}t\frac{2}{\sqrt{\pi}}\int_{0}^{+\infty}e^{-\theta^2}\,\mathrm{d}\theta\notag\\
&=\int_{-\infty}^{+\infty}x(t)e^{-j2\pi ft}\,\mathrm{d}t\notag\\
&=X(f)
\end{align}
\subsection{表格}
\begin{table}[H]
	\begin{center}
		\caption{希腊字母表\label{tab:Greek}}
		\begin{tabular}{|c|c|c|c|c|}
			\hline
			Alpha & Beta & Gamma & Delta & Theta\\
			\hline
			$\alpha$ & $\beta$ & $\gamma$ & $\delta$ & $\theta$\\
			\hline
			$A$ & $B$ & $\Gamma$ & $\Delta$ & $\Theta$\\
			\hline
		\end{tabular}
		\end{center}
\end{table}
这是对表\ref{tab:Greek}的引用

\begin{table}[htbp]
	\caption{不同电力系统频率测量算法时间复杂度比较}\label{table2:1}
	\centering\zihao{5}
	\begin{tabular}{cccc}
		\toprule[1.5pt]
		算法 &  加法 & 乘法 & 时间复杂度 \\
		\midrule[1pt]
		TQDS       & $QN^2+QN/2+Q+1$        &$QN^2$                      & $O(N^2)$     \\
		WIFFT         & $(QN+1)\log_2(QN+1)$ & $(QN+1)*(1+\log_2(QN+1))$ &$O(N\log_2N)$\\
		本章算法 &$3(QN+1)\log_2(QN+1)$    & $(QN+1)(1+3\log_2(QN+1))$          & $O(N\log_2N)$\\
		\bottomrule[1.5pt]
	\end{tabular}
	\vspace{\baselineskip}
\end{table}

本章对时域准同步算法(Time Domain Quasi-synchronous,TQDS)、加窗插值~FFT~算法(Windowed Interpolated FFT,WIFFT)以及本章所提算法的时间复杂度进行分析。因~TQDS~需要进行迭代运算,故设总采样点数为~$QN+1$,其中~$Q$~为迭代次数,$N$~为单次迭代所需的数据点长度。TQDS~共需要~$QN^2$~次加法和~$QN^2+QN/2+Q+1$~次乘法,因此~TQDS~的时间复杂度为~$O(N^2)$。WIFFT~的计算量主要为~FFT~运算,共需进行~$(QN+1)\log_2(QN+1)$~次加法和~$(QN+1)(1+\log_2(QN+1))$~次乘法,因此~WIFFT~的时间复杂度为~$O(N\log_2N)$。对于本章所提出的算法,由于线性卷积运算采用快速卷积来进行计算,因此共需进行~$3(QN+1)\log_2(QN+1)$~加法和~$(QN+1)(1+3\log_2(QN+1))$~次乘法,算法时间复杂度为~$O(N\log_2N)$。表~\ref{table2:1}~对三种频率测量算法的时间复杂度进行了对比。由表~\ref{table2:1}~可见,TQDS~的时间复杂度比其它两种算法要高,本章算法和~WIFFT~时间复杂度相当,有利于算法的实时实现。
\subsection{图形}
这个示例为插入图片:
\begin{figure}[H]
	\centering
		\includegraphics[width=0.618\textwidth]{figure.jpg}%图片名称,放在/figures目录下
	\caption{图片插入\label{fig:fig}}
\end{figure}
具体代码:
\begin{verbatim}%抄写环境
\begin{figure}[H]
\centering
%图片放在/figures目录下
\includegraphics[width=0.618\textwidth]{figure.jpg}
\caption{图片插入\label{fig:fig}}
\end{figure}
\end{verbatim}
\begin{figure}[H]
\centering
		\includegraphics[width=0.618\textwidth]{QHU.bmp}
\caption{青海大学\label{fig:qhu}}
\end{figure}
对于图\ref{fig:fig}和图\ref{fig:qhu}的引用。

这个示例为插入图片添加双语题注实例:
\begin{figure}[H]
	\centering
	\includegraphics[width=0.3\textwidth]{青大老化院.jpg}  %图片名称,放在/figures目录下
	\bicaption{青大老化院}{QHU's old chemical engineering institute}
\end{figure}

\subsection{引用}
\subsubsection{交叉引用}
对所有需要引用的公式、表格、图形,执行插入--标签后,即可使用插入-- 交叉引用自动产生引用。
\begin{itemize}
\item 哈密顿量见方程~\eqref{eq:sbm}。
\item 希腊字母表见表~\ref{tab:Greek}。引用格式与方程引用格式不同
\item 校名标准字如图~\ref{fig:qhu}。 引用格式与方程引用格式不同
\end{itemize}
具体见代码:
\begin{verbatim}
\begin{itemize}
\item 哈密顿量见方程~\eqref{eq:sbm}。
\item 希腊字母表见表~\ref{tab:Greek}。引用格式与方程引用格式不同
\item 校名标准字如图~\ref{fig:qhu}。 引用格式与方程引用格式不同
\end{itemize}
\end{verbatim}
\subsubsection{文献引用}
将引文的bib数据库(默认文件名为reference.bib)放入模板根目录下的references文件夹,即可通过插入--文献引用自动产生引文。
\begin{itemize}
\item Journal:An article \cite{ELIDRISSI94,MELLINGER96,SHELL02,cnarticle}。
\item Book:An book \cite{IEEE-1363,tex,companion}。
\item Conference:A conference \cite{kocher99,DPMG,cnproceed}。
\item Manual:A manual\cite{NPB2}.
\item MasterThesis:\cite{zhubajie,metamori2004,shaheshang,FistSystem01}.
\end{itemize}
\subsection{伪代码实现}
\begin{algorithm}
\caption{放进冰箱的大象}\label{算法实例}
\begin{algorithmic}
	\REQUIRE 有一只大象
	\ENSURE 放进冰箱里
	\FOR {没有剩余的大象}
	\IF {大象比冰箱大}
	\STATE 把大象分割
	\ENDIF
	\ENDFOR
	\STATE 第一步
	\STATE 第二步
	\STATE 第三步
\end{algorithmic}
AAA\end{algorithm}
\subsection{代码展示}
可以把你的程序添加到附录里,展示自己的工作。
\begin{lstlisting}[language={[ANSI]C}, numbers=left]
#include <stdio.h>
int main(int argc, char ** argv)
{
/*打印Hello,world*/
printf("Hello, world!\n");

return 0;
}
\end{lstlisting}
\section{依赖}
QHUthesis依赖于以下宏包,这些宏包在常见的\LaTeX{}发行版中都包括,在安装使用之前,请确定你的\TeX{}发行版中都已正常安装这些宏包
\begin{table}[H]
	\centering
	\begin{tabular}{cccc}
		\hline
		{footmisc} &  {amsmath} &  {amsfonts} &  {amssymb} \\
		
		{graphicx} &  {svgnames} &  {xcolor} &  {mathptmx} \\
		
		{float} &  {fontenc} &  {fancyhdr} &  {lastpage} \\
		
		{etoolbox} &  {fancy} &  {caption} &  {array} \\
		
		{makecell} &  {forloop} &  {xstring} &  {hyperref} \\
		
		{tabularx} &  {enumitem} &  {ntheorem} &  {algorithm}\\
		
		{algorithmic} &  {bibentry} &  {xeCJK} &  {CJK} \\
		{listings} &  {courier} &  {} &  {} \\
		\hline
	\end{tabular}
\end{table}
如果你尚未安装这些宏包,可以启动你的 \TeX{} 发行版的宏包管理器
来安装;或者到 \url{http://www.ctan.org} 上搜索下载并安装。
\section{基本设置}
\begin{enumerate}
	\item 图片搜索路径默认设置为模板根目录下的figures/。
	\item bib数据库默认设置为模板根目录下的references/reference.bib。 其中bib文件可由任意文献库管理软件自动生成
\end{enumerate}




